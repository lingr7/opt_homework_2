% !Mode:: "TeX:UTF-8"
% !TEX program  = xelatex

% 模板名称:YangThesis
% 模板版本:V1.0
% 模板作者:JingXuan Yang
% 联系作者:yangjingxuan@stu.hit.edu.cn
% 模板来源:数模竞赛模板的二次开发
% 模板适用:普通课程论文,略作修改可用作毕业设计(论文)
% 模板编译:XeLaTeX,编译两次,两次,两次!!!
% 更新时间:4/14/2019
% 模板说明:本模板尽量把接口留在了类文件外面,但是顾名思义,模板即为
%                 部分样式的集合,有些样式是定义在类文件里面的,在外面无法
%                 修改。类文件中注释齐全,很多样式可以个性化定义,建议拥有
%                 一定LaTeX基础以后再尝试修改类文件(YangThesis.cls)。

% 本模板提供有、无封面两种选择
% 1、采用封面
% 模板默认采用封面
% no-math选项指不采用模板文件对数学符号字体的定义
% 可在settings.tex里设置数学符号字体

\documentclass[no-math]{YangThesis} 

% 2、去掉封面,添加<withoutpreface>选项
%\documentclass[no-math, withoutpreface]{YangThesis} 

% 导入配置文件settings.tex,
% 配置参数均存储在settings.tex文件中,
% 添加或修改均需在该文件中进行
\input{settings.tex}

% 封面信息
% 论文类别
\papercategory{最优化理论与应用课程报告}
% 论文标题
\title{拟牛顿法实验报告}

% 学校名称
\schoolname{南京航空航天大学}
% 学院名称
\departname{计算机科学与技术学院}
% 专业
\majorin{计算机技术}
% 班级
%\classnumber{中特社17班}
% 姓名
\authorname{林国瑞}
% 学号
\studentID{SF1916009}
% 教师
\teacher{朱琨}
% 日期
\dateinput{2019年1月10日}

%%%=====================%%%
% 开始写文章
\begin{document}

% 生成标题
\maketitle

% 页码从1开始计数
\setcounter{page}{1}
% 页码采用罗马数字格式
\pagenumbering{Roman}

\vspace{-1.3cm}

% 生成目录
\tableofcontents
% 将目录添加到目录中
\addcontentsline{toc}{section}{目\,\,录}


% 分页,撰写正文
\clearpage
% 调整标题与上边的距离
\vspace{-1cm}
% 第1章的标题
\section{什么是科学}
% 关于论文的主题,基本上就是人工智能和伪科学二选一了,可是,伪科学,迷信,这些内容也已经有不少好的材料了。这种人文学科的问题可能就是值得发表的东西不多吧。何况已发表的也就是那个样子。不容易做出成果来。基本上,我总结一下看到的几篇网页就行了吧。
	
% 基本上所有搜集材料的过程结束了,目前不再在网上找阅读材料了,已经足够多了。

% 页码从1开始计数
\setcounter{page}{1}
% 页码格式采用阿拉伯数字
\pagenumbering{arabic}
\subsection{前提知识}





% 正文内容,注意LaTeX分段有两种方法,直接空一行或者使用<\par>
% 默认首行缩进,不需要在代码编辑区手动敲空格

% 需手动分页,为实现分页与不分页均可
% \newpage

% Create LaTeX tables online  http://www.tablesgenerator.com/#
% 这个网站比较神奇,竟然还有文本格式的表格生成,不过从使用角度讲,markdown语法太好用了,估计平常用的格式就是markdown,今天认识到了,日常的wps可以用来画图,可以用来画自己需要的示意图。通过表格的方法,画出时间线图,既能示意,效率也很高。比琢磨半天的xmind清晰明了多了。
% Please add the following required packages to your document preamble:
% \usepackage{booktabs}
% \usepackage{graphicx}





\begin{figure}
	\centering
	\includegraphics[width=0.7\linewidth]{DFP_quasi_newton}
	\caption{}
	\label{fig:dfpquasinewton}
\end{figure}



\begin{appendix}
	Program List\\
	1.The code used to make the area chart
	%\textbf{\textcolor[rgb]{0.98,0.00,0.00}{Input matlab source:}}
	\lstinputlisting[language=Matlab]{mcode/bfgs.m}

\end{appendix}


\newpage
\addcontentsline{toc}{section}{参考文献}
%\bibliography{opt}
\printbibliography[heading=bibliography,title=参考文献]
\end{document} 